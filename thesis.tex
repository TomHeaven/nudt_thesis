%%
%% This is file `thesis.tex',
%% generated with the docstrip utility.
%%
%% The original source files were:
%%
%% nudtpaper.dtx  (with options: `thesis')
%% 
%% This is a generated file.
%% 
%% Copyright (C) 2018 by TomHeaven <hanlin_tan@nudt.edu.cn>
%% 
%% This file may be distributed and/or modified under the
%% conditions of the LaTeX Project Public License, either version 1.3a
%% of this license or (at your option) any later version.
%% The latest version of this license is in:
%% 
%% To produce the documentation run the original source files ending with `.dtx'
%% through LaTeX.
%% 
%% Thanks LiuBenYuan <liubenyuan@gmail.com> for maintainence.
%% Thanks Xue Ruini <xueruini@gmail.com> for the thuthesis class!
%% Thanks sofoot for the original NUDT paper class!
%% 
%1. 规范硕士导言
% \documentclass[master,ttf]{nudtpaper}
%2. 规范博士导言
% \documentclass[doctor,twoside,ttf]{nudtpaper}
%3. 如果使用是Vista
% \documentclass[master,ttf,vista]{nudtpaper}
%4. 建议使用ttf字体
% \documentclass[doctor,twoside,fz]{nudtpaper}
%5. 如果想生成盲评,传递anon即可,仍需修改个人成果部分
% \documentclass[master,otf,anon]{nudtpaper}
%
\documentclass[doctor,ttf,twoside]{nudtpaper}
\usepackage{mynudt}
\usepackage{multirow,array}

\classification{TP391}
\serialno{1605xxxx}
\confidentiality{公开}
\UDC{681}
\title{基于先验模型与深度学习的图像去噪\\方法研究}
\displaytitle{基于先验模型与深度学习的图像去噪方法研究}
\author{谭同学}
\zhdate{\zhtoday}
\entitle{Joint Demosaicing and Denoising of Bayer Images}
\enauthor{Tom Heaven}
\endate{\entoday}
\subject{控制科学与工程}
\ensubject{Control Science and Engineering}
\researchfield{多媒体信息系统与虚拟现实技术}
\supervisor{张XX\quad{}教授}
\cosupervisor{}  % 协助指导教师,没有就空着
\ensupervisor{Prof. XX Zhang}
\encosupervisor{} % 协助指导教师英文,没有就空着
\papertype{工学}
\enpapertype{Engineering}
% 加入makenomenclature命令可用nomencl制作符号列表。

\begin{document}
	\graphicspath{{figures/}}
	% 制作封面,生成目录,插入摘要,插入符号列表 \\
	% 默认符号列表使用denotation.tex,如果要使用nomencl \\
	% 需要注释掉denotation,并取消下面两个命令的注释。 \\
	% cleardoublepage% \\
	% printnomenclature% \\
\maketitle
\frontmatter
\tableofcontents
\listoftables
\listoffigures

\midmatter
\input{data/abstract}
\input{data/denotation}

%书写正文,可以根据需要增添章节; 正文还包括致谢,参考文献与成果
\mainmatter
\renewcommand\UrlFont{\timesnr}
\makeatletter
\newcounter{blankpages}
\def\cleardoublepage{%
	\clearpage
	\if@twoside
	\ifodd\c@page
	\else
	\hbox{}\newpage\stepcounter{blankpages}%
	\thispagestyle{empty}%
	\if@twocolumn\hbox{}\newpage\fi
	\fi
	\fi
}
\newcommand{\@romannoblank}[1]{%
	\@roman{\numexpr#1-\value{blankpages}\relax}%
}
\makeatother

\chapter{第一章题目}

本章的主要内容与学校提供的Word模板中内容一致,图片与表格均采用原始设定大小,%
主要是为了说明格式的统一。%
但是,\LaTeX{}的一些禁则,专业排版的能力,对公式及文献的处理都是得天独厚的,%
我们不必刻意去追求与Word的完美匹配。而且你将会发现,用\LaTeX{}书写论文的美! %

\section{(1.1 题目)}
正文内容

\subsection{(1.1.1 题目)}
正文内容

正文内容

\begin{figure}[htp]
	\centering
	\includegraphics{picmain}
	\caption{图 1.1 名称}
\end{figure}

\subsubsection{(1.1.1.1 题目)}
正文内容

正文内容

正文内容

\subsubsection{(1.1.1.2 题目)}
正文内容

正文内容

正文内容

\subsection{(1.1.2 题目)}
正文内容

正文内容

\begin{figure}[htp]
	\centering
	\includegraphics{picmain}
	\caption{图 1.2 名称}
\end{figure}

\section{(1.2 题目)}
正文内容

正文内容

\begin{table}[htp]
	\centering
	\begin{minipage}[t]{0.8\linewidth} % 如果想在表格中使用脚注,minipage是个不错的办法
		\caption[表 1.1 名称]{}
		\begin{tabular*}{\textwidth}{lp{10cm}}
			\toprule[1.5pt]
			{\hei 列1} & {\hei 列2} \\
			\midrule[1pt]
			&  \\
			& \\
			& \\
			& \\
			& \\
			& \\
			\bottomrule[1.5pt]
		\end{tabular*}
	\end{minipage}
\end{table}

正文内容

正文内容

正文内容

正文内容

\section{(1.3 题目)}
正文内容

正文内容

正文内容

正文内容

正文内容

正文内容

\subsection{(1.3.1 题目)}
正文内容

\begin{figure}[htp]
	\centering
	\includegraphics{picmain}
	\caption{图 1.3 名称}
\end{figure}

\subsection{(1.3.2 题目)}
正文内容

正文内容

\begin{table}[htp]
	\centering
	\begin{minipage}[t]{0.8\linewidth} % 如果想在表格中使用脚注,minipage是个不错的办法
		\caption[表 1.2 名称]{}
		\begin{tabular*}{\textwidth}{lp{10cm}}
			\toprule[1.5pt]
			{\hei 列1} & {\hei 列2} \\
			\midrule[1pt]
			&  \\
			& \\
			& \\
			& \\
			& \\
			& \\
			\bottomrule[1.5pt]
		\end{tabular*}
	\end{minipage}
\end{table}


\input{data/chap02}

\input{data/ack}

\cleardoublepage
\phantomsection
\addcontentsline{toc}{chapter}{参考文献}
\bibliographystyle{bstutf8}
\bibliography{ref/refs}

\input{data/resume}
% 最后,需要的话还要生成附录,全文随之结束。
\appendix
\backmatter
\input{data/appendix01}

\end{document}
